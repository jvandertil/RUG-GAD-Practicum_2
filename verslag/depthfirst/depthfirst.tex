\chapter{Depth-first search}
\label{chap:depthfirst}

De eerste methode die onderzocht is voor het zoeken naar een augmented path is de Depth-first search methode. Deze methode zal, zoals de naam suggereert, de diepte in gaan op zoek naar $t$. 


Tijdens het zoeken naar $t$ worden de edges gelabeld met $discovery$, $unexplored$ en $back$. 

\subsection{Pseudocode}
De pseudocode waar de code op gebaseerd is, is te vinden in algoritme \ref{alg:DFS}.

\begin{algorithm}[h]
\caption{Depth-first search Algorithm}
\label{alg:DFS}
\begin{algorithmic}
\REQUIRE Input: Graph g, Start vertex s, End vertex t, HashMap parents with vertexes and its parent edges
\STATE Label s as $EXPLORED$
\FORALL{edge $e \in s.incidentEdges$}
\IF e is not labeled as $UNEXPLORED$ $\and$ s.residualCapacity(e) > 0
\STATE w $\gets $ g.opposite(s, e)
\IF w is labeled as $UNEXPLORED$
\STATE label $e$ as $DISCOVERY$ edge
\STATE set $e$ as parent of $w$ in the hashmap parents
\STATE recursive call with g, w, t and parents
\ELSE
\STATE label $e$ as $BACK$ edge
\ENDIF
\ENDIF
\ENDFOR
\end{algorithmic}
\end{algorithm}

Wanneer het eindpunt $t$ bereikt is kan het algoritme stoppen. Nu kan met behulp van de $parents$ gezocht worden naar een pad van $s$ naar $t$ door te kijken wat de parent edge $e$ is van $t$. Nu zal gekeken worden naar de parent edge van de overstaande van $t$ via edge $e$. Door dit te doen tot er geen parent edge is zal $s$ bereikt worden.