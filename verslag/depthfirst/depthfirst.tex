\chapter{Depth-first search}
\label{chap:depthfirst}

De eerste methode die onderzocht is voor het zoeken naar een augmented path is de Depth-first search methode. Deze methode zal, zoals de naam suggereert, de diepte in gaan op zoek naar $t$. 

De recursieve methode DFS verwacht als invoer een graaf $g$, een start vertex $s$, een eind vertex $t$ en een map $parents$. Deze map wordt later gebruikt om de weg van $t$ naar $s$ te vinden. Bij elke aanroep zal $s$ gelabeld worden als $EXPLORED$. Hierna zullen alle aanliggende kanten van $s$ bijlangs gegaan worden om te kijken of hier nog een eventueel pad mogelijk is. Dit wordt gedaan door te kijken naar de overstaande knoop $w$ via $e$. Wanneer $w$ gelabeld is als $UNEXPLORED$ en de kant $e$ nog een capaciteit heeft, is hier een pad mogelijk. $e$ zal nu gezet worden als de parent van $w$ en daarnaast ook nog gelabeld worden als $DISCOVERY$. Nu zal een recursieve aanroep gedaan worden met de parameters respectievelijk $g$, $w$, $t$ en $parents$.
Als $w$ niet gelabeld is als $UNEXPLORED$ zal $e$ gelabeld worden als $BACK$.

\subsection{Pseudocode}
De pseudocode waar de code op gebaseerd is, is te vinden in algoritme \ref{alg:DFS}.

\begin{algorithm}[h]
\caption{Depth-first search Algorithm}
\label{alg:DFS}
\begin{algorithmic}
\REQUIRE Input: Graph g, Start vertex s, End vertex t, HashMap parents with vertexes and its parent edges
\STATE Label s as $EXPLORED$
\FORALL{edge $e \in s.incidentEdges$}
\IF e is not labeled as $UNEXPLORED$ && s.residualCapacity(e) > 0
\STATE w $\gets $ g.opposite(s, e)
\IF w is labeled as $UNEXPLORED$
\STATE label $e$ as $DISCOVERY$ edge
\STATE set $e$ as parent of $w$ in the hashmap parents
\STATE recursive call with g, w, t and parents
\ELSE
\STATE label $e$ as $BACK$ edge
\ENDIF
\ENDIF
\ENDFOR
\end{algorithmic}
\end{algorithm}

Wanneer het eindpunt $t$ bereikt is kan het algoritme stoppen. Nu kan met behulp van de $parents$ gezocht worden naar een pad van $s$ naar $t$ door te kijken wat de parent edge $e$ is van $t$. Nu zal gekeken worden naar de parent edge van de overstaande van $t$ via edge $e$. Door dit te doen tot er geen parent edge is zal $s$ bereikt worden.