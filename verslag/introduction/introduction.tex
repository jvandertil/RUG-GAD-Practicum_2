\chapter{Introductie}

Dit verslag maakt deel uit van de cursus Gevorderde Algoritmen en Datastructuren van de Rijksuniversiteit Groningen. 
In dit verslag zal de tweede practicum opdracht behandeld worden. 
Deze opdracht omvat het vinden van een maximum flow in een flow network en de verschillende manieren, om dit te doen, te analyseren.

In hoofdstuk \ref{chap:maxFlowProblem} zal het probleem van het vinden van een maximum flow en het gebruikte algoritme beschreven worden. De hoofdstukken \ref{chap:depthfirst}, \ref{chap:breadthfirst} \& \ref{chap:priorityfirst} beschrijven de algoritmen die gebruikt worden om een pad te vinden door het netwerk. De conclusies van dit onderzoek staan in hoofdstuk \ref{chap:conclusion}.

Omdat het Ford-Fulkerson algoritme niet aangeeft op welke manier er een 'augmenting path' gevonden dient te worden, zijn er meerdere methodes beschikbaar.
De methodes die onderzocht zullen worden in dit document zijn:

\begin{enumerate}
	\item Depth-first search;
	\item Breadth-first search;
	\item Priority-first search.
\end{enumerate}

In het geval van een breadth-first search is het algoritme ook bekend als het Edmonds-Karp algoritme.